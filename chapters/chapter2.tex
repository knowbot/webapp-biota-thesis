%!TEX root = ../dissertation.tex
\begin{savequote}[75mm]
This is some random quote to start off the chapter.
\qauthor{Firstname lastname}
\end{savequote}

\chapter{Rails 5}


I principi cardine di Rails sono due:
\begin{itemize}
    \item\textbf{Do not Repeat Yourself}: noto come principio DRY, sostiene che vadano evitate tutte le forme di ripetizione e ridondanza logica nell'implementazione del software, evitando duplicazione di codice e semplificando lo sviluppo.
    \item\textbf{Convention over Configuration}: noto come principio CoC, sostiene il programmatore dovrebbe dover esplicitare solo le parti ``non convenzionali'' del codice; ad esempio, se esiste \textit{per convenzione} una corrispondenza tra il nome di una classe e il nome di una tabella del database, essa non dovrebbe essere specificata.
\end{itemize}


\subsubsection{Model}
In Rails un \textit{model} è una classe associata ad una tabella del database secondo uno standard convenzionale: una tabella corrisponde una classe, le colonne sono convertite in attributi e le righe rappresentate come istanze della classe. I modelli possono venire generati automaticamente dalla definizione della tabella. Oltre ai normali vincoli di database, in Rails è possibile definire ulteriori controlli sui valori in database direttamente nel \textit{model}. La filosofia di Rails prevede che i \textit{model} contengano la quasi interezza della \textit{business logic}.

\subsubsection{Controller}
I \textit{controller} sono componenti che rispondo alle richieste del server web all'applicazione, determinando quale \textit{view} caricare; essi possono anche interrogare i \textit{model} per mostrare informazioni aggiuntive, e rendere disponibili "azioni" per fare richieste al server. I controller sono resi disponibili mediante il file di \textit{routing} \texttt{routes.rb}, in cui vengono associati a delle specifiche richieste; Rails incoraggia gli sviluppatori all'uso di \textit{RESTful routes}, che includono azioni come ``index'', ``new'', ``show'', ``edit''.

\subsubsection{View}
Di default le \textit{view} sono file con estensione \texttt{.erb} contenenti codice HTML misto a codice Ruby, che vengono elaborati a \textit{run-time} permettendo una visualizzazione dinamica delle informazioni. In alternativa, per esempio nell'implementazione di una RESTful API, una \textit{view}può essere un file JSON contenente il \textit{body} di una risposta.