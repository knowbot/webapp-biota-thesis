%!TEX root = ../dissertation.tex
\externaldocument{../frontmatter/abbr}
\externaldocument{./appendixA}
\begin{savequote}[75mm]
This is some random quote to start off the chapter.
\qauthor{Firstname lastname}
\end{savequote}
\chapter{Analisi dei requisiti}
\label{sec:chap3}
\section{Definizione dei requisiti}
Nella seguente sezione verranno esposti in via generale i requisiti che l'applicativo realizzato doveva soddisfare. È importante notare che, data la natura agile del metodo di lavoro impiegato dall'azienda, gli obiettivi presentati nel piano di lavoto sono stati soggetti a modifiche e aggiunte di lieve e media entità; per completezza e possibilità di confronto, di seguito verranno esposti sia i requisiti iniziali che quelli elaborati a fronte dell'interazione con il cliente.
\subsection{Notazione}
Si farà riferimento a requisiti ed obiettivi secondo le rispettive notazioni:
\begin{itemize}
    \item \textbf{O} indica un requisito obbligatorio, vincolante in quanto obiettivo primario richiesto dal committente.
    \item \textbf{D} indica un requisito desiderabile, non vincolante o strettamente necessario, ma dal riconoscibile valore aggiunto.
    \item \textbf{F} indica un requisito facoltativo, rappresentante un valore aggiunto non necessariamente competitivo.
\end{itemize}
\subsection{Piano di lavoro}
Il piano di lavoro, stilato in collaborazione con il relatore ed il tutor aziendale nella settimana precedente all'inizio dello stage, espone una serie di requisiti emersi dalle esigenze del cliente emerse in fase di avvio del progetto.
\subsubsection{Requisiti}
\paragraph{Obbligatori}
\begin{itemize}
    \item \underline{O01}: Analisi ed implementazione \ref{itm:api} \ref{itm:rest} laboratorio
    \item \underline{O02}: Analisi ed implementazione \ref{itm:api} GraphQL per il flusso operativo base
\end{itemize}
\vspace{-15pt}
\paragraph{Desiderabili}
\begin{itemize}
    \item \underline{D01}: Analisi ed implementazione di test dei sistemi di pagamento e fatturazione
    \item \underline{D02}: Suite di testing del software prodotto
    \item \underline{D03}: Documentazione completa
\end{itemize}
\vspace{-15pt}
\paragraph{Facoltativi}
\begin{itemize}
    \item \underline{F01}: Ulteriori modifiche all’applicazione che esulano da quando riportato nel piano di lavoro
\end{itemize}
\subsection{Modifiche al piano}
Durante il corso dello stage, nello svolgimento del ruolo di Product Owner da parte del tutor aziendale, ci sono state varie interazioni con il cliente che hanno portato a ridefinire alcuni requisiti desiderabili e a specificare in modo più preciso i requisiti facoltativi. Tali modifiche sono sempre state fatte in accordo alle possibilità di soddisfazione entro i tempi previsti, e tenendo in mente che la piattaforma sarebbe entrata in produzione al completamento delle funzionalità per l'esecuzione flusso di attività base.
\subsubsection{Requisiti rivisti}
\paragraph{Obbligatori}
\begin{itemize}
    \item \underline{O01}: Analisi ed implementazione \ref{itm:api} \ref{itm:rest} laboratorio
    \item \underline{O02}: Analisi ed implementazione \ref{itm:api} GraphQL per il flusso operativo base
    \item \underline{O03}: Implementazione delle strutture per il questionario
\end{itemize}
\vspace{-15pt}
\paragraph{Desiderabili}
\begin{itemize}
    \item \underline{D01}: Generazione PDF con informazioni specifiche
    \item \underline{D02}: Suite di testing del software prodotto
    \item \underline{D03}: Documentazione completa
    \item \underline{D04}: Registrazione del consenso al trattamento dei dati
\end{itemize}
\vspace{-15pt}
\paragraph{Facoltativi}
\begin{itemize}
    \item \underline{F01}: Autenticazione mediante HMAC
    \item \underline{F02}: Ulteriori modifiche all’applicazione
\end{itemize}
\section{Casi d'uso}
\label{sec:usecase}
Di seguito verranno descritti i casi d'uso dell'applicazione, per quanto riguarda le funzionalità da me progettate; non verranno quindi tenute in considerazione eventuali funzionalità pre-esistenti che fossero parte dell'ecosistema a cui appartiene la piattaforma.

I casi d'uso verranno inoltre considerati per quanto concerne il \textit{back end} dell'applicazione, ovvero sulla base delle funzionalità realizzate per essere utilizzate dal \textit{front end}: eventuali aggiunte, quali la visualizzazione di messaggi di errore dettagliati o reindirizzamenti a pagine differenti della piattaforma web, verranno ignorate.

I casi d'uso verranno descritti mediante dei diagrammi \ref{itm:uml} che rappresentano le possibili interazioni tra gli \textbf{attori} ed il \textbf{sistema} dal punto di vista dei primi. È importante notare che il sistema, ovvero il \textit{back end} sviluppato, non viene utilizzato direttamente dagli utenti finali ma risponde a richieste di servizi esterni come il \textit{front end} (con l'eccezione degli utenti amministrativi, a cui viene fornita un'interfaccia di interazione ad-hoc): i casi d'uso verranno dunque trattati di conseguenza.

Di seguito verranno elencati solo i casi d'uso principali; per quelli secondari consultare l'\hyperref[sec:AppendixA]{appendice A}.
\subsection{Attori}
\begin{itemize}
    \item \textbf{Utente non autenticato}: un utente che deve effettuare l'autenticazione per accedere alle funzionalità disponibili al suo ruolo;
    \item \textbf{Amministratore}: un utente autenticato con privilegi di accesso al pannello amministrativo.
    \item \textbf{Front end}: una istanza del \textit{front end} della piattaforma;
    \item \textbf{Gestionale laboratorio}: il software gestionale del laboratorio;
\end{itemize}
\subsection{Struttura}
Per ogni caso d'uso, identificato da un codice univoco, vengono specificati gli attori coinvolti, lo scopo da raggiungere, le azioni previste e le condizioni del sistema prima e dopo tali azioni. I casi d'uso verranno descritti dalla seguente struttura:
\begin{itemize}
    \item Codice identificativo: \textbf{UC \{codice\_padre\}.\{codice\_figlio\}}
        \vspace{-5pt}
        \begin{itemize}
            \item \textbf{UC} indica che si tratta di un caso d'uso;
            \item \textbf{codice\_padre} identifica il caso d'uso principale a cui si fa riferimento;
            \item \textbf{codice\_figlio} identifica il sottocaso specifico;
        \end{itemize}
        \vspace{-5pt}
    \item Titolo
    \item Diagramma UML
    \item Attori
    \item Attori secondari, se presenti
    \item Scopo e descrizione
    \item Precondizioni
    \item Scenario principale
    \item Postcondizioni
    \item Inclusioni (se presenti)
    \item Estensioni (se presenti)
\end{itemize}
\subsection{Casi d'uso generali}
Per una miglior suddivisione, le operazioni che avvengono sotto condizioni molto simili (principalmente quelle che possono essere richieste da un utente autenticato con un determinato ruolo) saranno raggruppate in dei casi d'uso di alto livello, d'ora in poi ``casi d'uso generali", identificati dalla lettera \textbf{G} prefissa al loro codice numerico. Tali casi d'uso servono solo per una miglior comprensione, quindi il loro codice non verrà considerato come \textbf{codice\_padre} nell'identificazione dei casi d'uso raggruppati.

\setlength{\parindent}{0cm}
\section{UC1: Autenticazione amministratore}
\label{sec:uc1}
\textsc{\textsc{\textbf{Attori:}}} Utente non autenticato.\\
\textsc{\textbf{Scopo e descrizione:}} L'attore vuole effettuare la procedura di autenticazione al pannello di amministrazione.\\
\textsc{\textsc{\textbf{Precondizioni:}}} L'applicativo è stato avviato con successo e l'attore ha accesso alla pagina di autenticazione.\\
\textsc{\textbf{Scenario principale:}}
\begin{itemize}
    \item L'attore inserisce un indirizzo email valido (\hyperref[sec:uc11]{UC 1.1});
    \item L'attore inserisce la password associata a tale indirizzo email (\hyperref[sec:uc12]{UC 1.2});
    \item L'attore clicca sul pulsante di \textit{login}.
\end{itemize}
\textsc{\textbf{Postcondizioni:}} L'attore viene autenticato dal sistema e ha accesso al pannello di amministrazione.\\
\textsc{\textbf{Estensioni:}} L'attore non viene autenticato e visualizza un messaggio di errore (\hyperref[sec:uc2]UC2{}).

\section{UC2: Errore autenticazione amministratore}
\label{sec:uc2}
\textsc{\textbf{Attori:}} Utente non autenticato.\\
\textsc{\textbf{Scopo e descrizione:}} L'attore vuole effettuare la procedura di autenticazione al pannello di amministrazione.\\
\textsc{\textsc{\textbf{Precondizioni:}}} L'attore ha inserito le proprie credenziali negli appositi campi.\\
\textsc{\textbf{Scenario principale:}}
\begin{itemize}
    \item L'attore clicca sul pulsante di \textit{login};
    \item L'attore visualizza un messaggio di errore che lo informa che le credenziali non sono corrette.
\end{itemize}
\textsc{\textbf{Postcondizioni:}} L'attore non viene autenticato dal sistema e rimane nella pagina di autenticazione.

\section{UC3: Autenticazione front end}
\label{sec:uc3}
\textsc{\textbf{Attori:}} Front end.\\
\textsc{\textbf{Scopo e descrizione:}} L'attore vuole effettuare la procedura di autenticazione per un utente della piattaforma.\\
\textsc{\textsc{\textbf{Precondizioni:}}} L'applicativo è stato avviato con successo e l'attore è collegato correttamente al sistema.\\
\textsc{\textbf{Scenario principale:}} L'attore invia una richiesta di autenticazione contenente le credenziali di accesso di un utente autorizzato.\\
\textsc{\textbf{Postcondizioni:}} L'utente specificato dall'attore viene autenticato, e l'attore ha ricevuto in risposta i parametri della relativa sessione attiva.\\
\textsc{\textbf{Estensioni:}}  L'utente specificato dall'attore non viene riconosciuto, e l'attore ha ricevuto in risposta un messaggio di errore che lo informa che le credenziali non sono corrette (\hyperref[sec:UC5]{UC5}).

\section{UC4: Reset password per accesso alla piattaforma}
\label{sec:uc4}
\textsc{\textbf{Attori:}} Front end.\\
\textsc{\textbf{Scopo e descrizione:}} L'attore vuole effettuare la procedura di reset della password per un utente della piattaforma.\\
\textsc{\textsc{\textbf{Precondizioni:}}} L'applicativo è stato avviato con successo e l'attore è collegato correttamente al sistema.\\
\textsc{\textbf{Scenario principale:}} L'attore invia una richiesta di reset della password di accesso di un utente autorizzato.\\
\textsc{\textbf{Postcondizioni:}} È stata generata ed inviata all'utente una nuova password,e l'attore ha ricevuto una risposta di conferma del successo dell'operazione.

\section{UC5: Errore di autenticazione front end}
\label{sec:UC5}
\textsc{\textbf{Attori:}} Front end.\\
\textsc{\textbf{Scopo e descrizione:}} L'attore vuole effettuare la procedura di autenticazione per un utente della piattaforma.\\
\textsc{\textsc{\textbf{Precondizioni:}}} L'attore ha inviato una richiesta di autenticazione al sistema.\\
\textsc{\textbf{Scenario principale:}} 
\begin{itemize}
    \item L'attore è in attesa di una risposta dal sistema;
    \item L'attore riceve in risposta un messaggio di errore che lo informa che le credenziali non sono corrette.
\end{itemize}
\textsc{\textbf{Postcondizioni:}} L'utente specificato dall'attore non viene riconosciuto e non ha ricevuto i dati di una sessione valida.

\section{UC G1: Operazioni amministrative}
\label{sec:ucg1}
Questo caso d'uso generale riassume le operazioni disponibili ad un utente autorizzato che abbia eseguito l'accesso al pannello di amministrazione. La maggior parte delle operazioni sui record sono identiche tra loro, e verranno quindi riportate in modo generalizzato; eventuali operazioni specifiche verranno contestualizzate nel proprio caso d'uso.
\textsc{\textbf{Attori:}} Amministratore.\\
\textsc{\textbf{Scopo e descrizione:}} L'attore vuole eseguire operazioni di amministrazione della piattaforma.\\
\textsc{\textsc{\textbf{Precondizioni:}}} L'attore ha ottenuto l'accesso al pannello di amministrazione.\\
\textsc{\textbf{Scenario principale:}} 
\begin{itemize}
    \item L'attore visualizza e gestisce una categoria di record (\hyperref[sec:UC6]{UC6});
    \item L'attore visualizza la \textit{dashboard} (\hyperref[sec:UC7]{UC7}).
    \item L'attore effettua la disconnessione dal pannello di amministrazione (\hyperref[sec:UC8]{UC8}).
\end{itemize}
\textsc{\textbf{Postcondizioni:}} L'attore ha correttamente portato a termine le operazioni desiderate.

\section{UC6: Gestione categoria di record}
\label{sec:UC6}
\textsc{\textbf{Attori:}} Amministratore.\\
\textsc{\textbf{Scopo e descrizione:}} L'attore vuole gestire una particolare categoria di record.\\
\textsc{\textsc{\textbf{Precondizioni:}}} L'attore è nella \textit{dashboard} del pannello di amministrazione.\\
\textsc{\textbf{Scenario principale:}} 
\begin{itemize}
    \item L'attore seleziona una scheda associato alla categoria desiderata e visualizza la tabella dei record.
    \item L'attore modifica l'ordine della tabella dei record (\hyperref[sec:UC61]{UC6.1});
    \item L'attore applica un filtro alla tabella dei record (\hyperref[sec:UC62]{UC6.2});
    \item L'attore rimuove un filtro dalla tabella dei record (\hyperref[sec:UC63]{UC6.3});
    \item L'attore aggiunge un nuovo record (\hyperref[sec:UC64]{UC6.4});
    \item L'attore visualizza un record particolare (\hyperref[sec:UC65]{UC6.5});
    \item L'attore modifica un record particolare (\hyperref[sec:UC66]{UC6.6});
    \item L'attore elimina un record particolare (\hyperref[sec:UC67]{UC6.7});
    \item L'attore gestisce un utente (\hyperref[sec:UC610]{UC6.10}).
    \item L'attore gestisce un cliente (\hyperref[sec:UC611]{UC6.11}).
\end{itemize}
\textsc{\textbf{Postcondizioni:}} L'attore ha eseguito tutte le operazioni di gestione della categoria di record selezionata con successo.

\section{UC7: Visualizzazione dashboard}
\label{sec:UC7}
\textsc{\textbf{Attori:}} Amministratore.\\
\textsc{\textbf{Scopo e descrizione:}} L'attore vuole visualizzare la \textit{dashboard} del pannello di amministrazione.\\
\textsc{\textsc{\textbf{Precondizioni:}}} L'attore sta visualizzando una pagina diversa dalla \textit{dashboard}.\\
\textsc{\textbf{Scenario principale:}} L'attore seleziona la scheda ``Dashboard".\\
\textsc{\textbf{Postcondizioni:}} L'attore viene reindirizzato alla pagina della \textit{dashboard}.

\section{UC8: Gestione profilo}
\label{sec:UC8}
\textsc{\textbf{Attori:}} Amministratore.\\
\textsc{\textbf{Scopo e descrizione:}} L'attore vuole gestire il proprio profilo utente.\\
\textsc{\textsc{\textbf{Precondizioni:}}} L'attore ha accesso al pannello di amministrazione.\\
\textsc{\textbf{Scenario principale:}}
\begin{itemize}
    \item L'attore clicca sull'icona del profilo.
    \item L'attore visualizza il record relativo al proprio profilo (\hyperref[sec:UC65]{UC6.5})
\end{itemize}
\textsc{\textbf{Postcondizioni:}} L'attore ha gestito correttamente il proprio profilo.

\section{UC9: Disconnessione}
\label{sec:UC9}
\textsc{\textbf{Attori:}} Amministratore.\\
\textsc{\textbf{Scopo e descrizione:}} L'attore vuole disconnettersi dal pannello di amministrazione.\\
\textsc{\textsc{\textbf{Precondizioni:}}} L'attore ha accesso al pannello di amministrazione.\\
\textsc{\textbf{Scenario principale:}} L'attore clicca sull'icona di disconnessione.\\
\textsc{\textbf{Postcondizioni:}} L'attore viene correttamente disconnesso dal pannello di amministrazione.

\section{UCG2: Operazioni per utente ``pharmacy''}
\label{sec:UCG2}
Questo caso d'uso generale riassume le operazioni disponibili al \textit{front end} quando è attiva una sessione per un utente con ruolo di ``pharmacy''. Come anticipato, le operazioni vengono considerate non dal punto di vista dell'utente finale della piattaforma, cioè il farmacista o cliente che si interfaccia con il \textit{front end}, ma come richieste che l'istanza di \textit{front end} può effettuare alle \ref{itm:api} del servizio \textit{back end}. Non verranno, per questo motivo, quindi fatte distinzioni tra interazioni che verrebbero effettuate da un farmacista o da un cliente. Infine, va considerato implicito che ognuna delle seguenti operazioni è ristretta ai record del database resi visibili dalla sessione attiva (nella fattispecie i record associati alla farmacia a cui appartiene l'utente autenticato).\\
\textsc{\textbf{Attori:}} Front end.\\
\textsc{\textbf{Scopo e descrizione:}} L'attore vuole eseguire operazioni permesse ad un utente con ruolo di ``pharmacy''.\\
\textsc{\textsc{\textbf{Precondizioni:}}} L'attore è collegato al sistema e possiede una sessione attiva per un utente con ruolo ``pharmacy''.\\
\textsc{\textbf{Scenario principale:}} 
\begin{itemize}
    \item L'attore modifica i dati della farmacia (\hyperref[sec:UC10]{UC10});
    \item L'attore gestisce gli utenti con ruolo ``pharmacy'' (\hyperref[sec:UC11]{UC11});
    \item L'attore gestisce gli esami (\hyperref[sec:UC12]{UC12});
    \item L'attore gestisce i clienti (\hyperref[sec:UC13]{UC13});
    \item L'attore gestisce le spedizioni (\hyperref[sec:UC14]{UC14});
    \item L'attore visualizza il referto specifico per un esame (\hyperref[sec:UC15]{UC15});
    \item L'attore effettua la disconnessione dalla piattaforma (\hyperref[sec:UC18]{UC18}).
\end{itemize}
\textsc{\textbf{Postcondizioni:}} L'attore ha correttamente portato a termine le operazioni desiderate.

\section{UC10: Aggiornamento dati della farmacia}
\label{sec:UC10}
\textsc{\textbf{Attori:}} Front end.\\
\textsc{\textbf{Scopo e descrizione:}} L'attore vuole aggiornare i dati della farmacia corrente.\\
\textsc{\textsc{\textbf{Precondizioni:}}} L'attore è collegato al sistema e possiede una sessione attiva per un utente con ruolo ``pharmacy''.\\
\textsc{\textbf{Scenario principale:}} L'attore invia una richiesta di aggiornamento del record della farmacia con i nuovi dati, potenzialmente allegando un'immagine da sostituire al logo corrente.\\
\textsc{\textbf{Postcondizioni:}} Il record corrispondente alla farmacia viene aggiornato e l'attore ha ricevuto una risposta che conferma il successo dell'operazione.

\section{UC11: Gestione utenti}
\label{sec:UC11}
\textsc{\textbf{Attori:}} Front end.\\
\textsc{\textbf{Scopo e descrizione:}} L'attore vuole gestire gli altri utenti con ruolo ``pharmacy'' appartenenti alla farmacia corrente.\\
\textsc{\textsc{\textbf{Precondizioni:}}} L'attore è collegato al sistema e possiede una sessione attiva per un utente con ruolo ``pharmacy''.\\
\textsc{\textbf{Scenario principale:}}
\begin{itemize}
    \item L'attore richiede la lista degli utenti (\hyperref[sec:UC111]{UC11.1});
    \item L'attore richiede i dati di un utente (\hyperref[sec:UC112]{UC11.2});
    \item L'attore richiede il reset della password di un utente (\hyperref[sec:UC113]{UC11.3});
    \item L'attore imposta una nuova password per un utente  (\hyperref[sec:UC114]{UC11.4});
    \item L'attore crea un utente (\hyperref[sec:UC115]{UC11.5});
    \item L'attore elimina un utente (\hyperref[sec:UC116]{UC11.6});
    \item L'attore aggiorna i dati di un utente (\hyperref[sec:UC117]{UC11.7}).
\end{itemize}
\textsc{\textbf{Postcondizioni:}} L'attore ha portato a termine con successo le operazioni di gestione degli utenti.

\section{UC12: Gestione esami}
\label{sec:UC11}
\textsc{\textbf{Attori:}} Front end.\\
\textsc{\textbf{Scopo e descrizione:}} L'attore vuole gestire gli altri utenti con ruolo ``pharmacy'' appartenenti alla farmacia corrente.\\
\textsc{\textsc{\textbf{Precondizioni:}}} L'attore è collegato al sistema e possiede una sessione attiva per un utente con ruolo ``pharmacy''.\\
\textsc{\textbf{Scenario principale:}}
\begin{itemize}
    \item L'attore richiede lo storico degli esami (\hyperref[sec:UC121]{UC12.1});
    \item L'attore richiede lo storico degli esami in base allo stato di avanzamento (\hyperref[sec:UC122]{UC12.2});
    \item L'attore richiede i dati un esame (\hyperref[sec:UC123]{UC12.3});
    \item L'attore richiede i dati di esame in base al codice campione (\hyperref[sec:UC124]{UC12.4});
    \item L'attore crea un nuovo esame (\hyperref[sec:UC125]{UC12.5});
    \item L'attore rimuove un esame (\hyperref[sec:UC126]{UC12.6});
    \item L'attore aggiorna i dati di un esame (\hyperref[sec:UC127]{UC12.7});
    \item L'attore aggiorna le risposte al questionario di un'esame (\hyperref[sec:UC128]{UC12.8}).
\end{itemize}
\textsc{\textbf{Postcondizioni:}} L'attore ha portato a termine con successo le operazioni di gestione degli utenti.

\section{UC13: Gestione clienti}
\label{sec:UC13}
\textsc{\textbf{Attori:}} Front end.\\
\textsc{\textbf{Scopo e descrizione:}} L'attore vuole gestire i clienti appartenenti alla farmacia corrente.\\
\textsc{\textsc{\textbf{Precondizioni:}}} L'attore è collegato al sistema e possiede una sessione attiva per un utente con ruolo ``pharmacy''.\\
\textsc{\textbf{Scenario principale:}}
\begin{itemize}
    \item L'attore richiede il registro dei clienti (\hyperref[sec:UC131]{UC13.1});
    \item L'attore richiede i dati di un cliente (\hyperref[sec:UC132]{UC13.2});
    \item L'attore crea un cliente (\hyperref[sec:UC133]{UC13.3});
    \item L'attore elimina un cliente (\hyperref[sec:UC134]{UC13.4});
    \item L'attore aggiorna i dati di un cliente (\hyperref[sec:UC135]{UC13.5});
    \item L'attore invia il codice per accettare la \textit{privacy policy} ad un cliente (\hyperref[sec:UC136]{UC13.6});
    \item L'attore verifica l'accettazione della \textit{privacy policy} per un cliente (\hyperref[sec:UC137]{UC13.7});
    \item L'attore effettua il caricamento di un modulo per il consenso alla \textit{privacy policy} firmato da un cliente (\hyperref[sec:UC138]{UC13.8}).
\end{itemize}
\textsc{\textbf{Postcondizioni:}} L'attore ha portato a termine con successo le operazioni di gestione dei clienti.

\section{UC14: Gestione spedizioni}
\label{sec:UC14}
\textsc{\textbf{Attori:}} Front end.\\
\textsc{\textbf{Scopo e descrizione:}} L'attore vuole gestire le spedizioni per la farmacia corrente.\\
\textsc{\textsc{\textbf{Precondizioni:}}} L'attore è collegato al sistema e possiede una sessione attiva per un utente con ruolo ``pharmacy''.\\
\textsc{\textbf{Scenario principale:}}
\begin{itemize}
    \item L'attore visualizza la lista delle spedizioni;
    \item L'attore visualizza una spedizione;
    \item L'attore crea una spedizione;
    \item L'attore elimina una spedizione;
    \item L'attore modifica i dati di una spedizione;
    \item L'attore visualizza la lettera di vettura per una spedizione;
    \item L'attore prenota il ritiro di una spedizione.
\end{itemize}
\textsc{\textbf{Postcondizioni:}} L'attore ha portato a termine con successo le operazioni di gestione delle spedizioni.

\section{UC15: Generazione referto specifico}
\label{sec:UC15}
\textsc{\textbf{Attori:}} Front end.\\
\textsc{\textbf{Scopo e descrizione:}} L'attore vuole visualizzare il referto di un esame, completo delle informazioni specifiche risultate dall'analisi e delle informazioni personali del cliente.\\
\textsc{\textsc{\textbf{Precondizioni:}}} L'attore è collegato al sistema ed l'esame desiderato è arrivato correttamente in stato \textbf{final}.\\
\textsc{\textbf{Scenario principale:}}
\begin{itemize}
    \item L'attore richiede la visualizzazione del referto, fornendo il codice identificativo interno dell'esame cui è associato;
    \item Il sistema processa il file PDF del referto inviato dal laboratorio aggiungendo la diagnosi del gastroenterologo e le generalità del paziente;
    \item L'attore riceve dal sistema il referto generato.
\end{itemize}
\textsc{\textbf{Postcondizioni:}} L'attore ha visualizzato correttamente il referto specifico.\\
\textsc{\textbf{Estensioni:}} L'attore invia una richiesta non valida (\hyperref[sec:UC16]{UC16}).

\section{UC16: Invio di una richiesta invalida}
\label{sec:UC16}
\textsc{\textbf{Attori:}} Front end.\\
\textsc{\textbf{Scopo e descrizione:}} L'attore vuole inviare una richiesta al sistema.\\
\textsc{\textsc{\textbf{Precondizioni:}}} L'attore è collegato al sistema e possiede una richiesta contenente dati non validi.\\
\textsc{\textbf{Scenario principale:}}
\begin{itemize}
    \item L'attore invia la richiesta con dati non validi;
    \item L'attore riceve un messaggio di errore che descrive la violazione commessa.
\end{itemize}
\textsc{\textbf{Postcondizioni:}} La richiesta dell'attore viene rifiutata e l'attore ha ricevuto un messaggio di errore appropriato.

\section{UC17: Superamento tentativi di invio del codice OTP}
\label{sec:UC17}
\textsc{\textbf{Attori:}} Front end.\\
\textsc{\textbf{Scopo e descrizione:}} L'attore vuole inviare il codice per l'accettazione della privacy policy ad un cliente della farmacia corrente.\\
\textsc{\textsc{\textbf{Precondizioni:}}} Sono già stati effettuati 3 tentativi nelle ultime 24 ore.\\
\textsc{\textbf{Scenario principale:}}
\begin{itemize}
    \item L'attore richiede l'invio del codice;
    \item L'attore riceve un messaggio di errore che lo informa del superamento del numero massimo di tentativi per la giornata.
\end{itemize}
\textsc{\textbf{Postcondizioni:}} La richiesta dell'attore viene rifiutata e l'attore ha ricevuto un messaggio di errore appropriato.

\section{UC18: Disconnessione dalla piattaforma}
\label{sec:UC18}
\textsc{\textbf{Attori:}} Front end.\\
\textsc{\textbf{Scopo e descrizione:}} L'attore vuole disconnettersi dalla piattaforma.\\
\textsc{\textsc{\textbf{Precondizioni:}}} L'attore è collegato al sistema e possiede una sessione attiva.\\
\textsc{\textbf{Scenario principale:}}
\begin{itemize}
    \item L'attore richiede la disconnessione dalla piattaforma;
    \item Il sistema invalida i parametri della sessione corrente.
\end{itemize}
\textsc{\textbf{Postcondizioni:}} L'attore viene disconnesso correttamente e il sistema non accetta più richieste con i parametri della sessione invalidata.

\section{UCG3: Operazioni per utente ``gastroenterologist''}
\label{sec:UCG3}
Questo caso d'uso generale riassume le operazioni disponibili al \textit{front end} quando è attiva una sessione per un utente con ruolo di ``gastroenterologist''. Valgono le stesse considerazioni fatte per \hyperref[sec:UCG2]{UCG2}, con l'eccezione dei record disponibili: nella descrizione dei seguenti casi d'uso le operazioni non sono ristrette ai record associati ad una farmacia, ma escludono rigorosamente i dati personali dei clienti.\\
\textsc{\textbf{Attori:}} Front end.\\
\textsc{\textbf{Scopo e descrizione:}} L'attore vuole eseguire operazioni permesse ad un utente con ruolo di ``gastroenterologist''.\\
\textsc{\textbf{Precondizioni:}} L'attore è collegato al sistema e possiede una sessione attiva per un utente con ruolo ``gastroenterologist''.\\
\textsc{\textbf{Scenario principale:}} 
\begin{itemize}
    \item L'attore richiede la lista di esami in attesa di diagnosi (\hyperref[sec:UC19]{UC19});
    \item L'attore richiede i dati di un esame in attesa di diagnosi (\hyperref[sec:UC20]{UC20});
    \item L'attore inserisce una diagnosi per un esame (\hyperref[sec:UC21]{UC21});
    \item L'attore inserisce la tabella della dieta per un esame (\hyperref[sec:UC22]{UC22});
    \item L'attore effettua la disconnessione dalla piattaforma (\hyperref[sec:UC18]{UC18}).
\end{itemize}
\textsc{\textbf{Postcondizioni:}} L'attore ha correttamente portato a termine le operazioni desiderate.

\section{UC19: Richiesta lista esami in attesa di diagnosi}
\label{sec:UC19}
\textsc{\textbf{Attori:}} Front end.\\
\textsc{\textbf{Scopo e descrizione:}} L'attore vuole ottenere la lista degli esami per cui è disponibile il referto delle analisi e sono quindi in attesa di diagnosi.\\
\textsc{\textsc{\textbf{Precondizioni:}}} L'attore è collegato al sistema e possiede una sessione attiva per un utente con ruolo ``gastroenterologist''.\\
\textsc{\textbf{Scenario principale:}} L'attore invia una richiesta per ottenere la lista di esami desiderata.\\
\textsc{\textbf{Postcondizioni:}} L'attore ha ricevuto correttamente i dati richiesti al sistema.\\
\textsc{\textbf{Estensioni:}} L'attore invia una richiesta non valida (\hyperref[sec:UC16]{UC16}).

\section{UC20: Richiesta informazioni per un esame in attesa di diagnosi}
\label{sec:UC20}
\textsc{\textbf{Attori:}} Front end.\\
\textsc{\textbf{Scopo e descrizione:}} L'attore vuole ottenere i dati di un esame  per cui è disponibile il referto delle analisi ed è quindi in attesa di diagnosi.\\
\textsc{\textsc{\textbf{Precondizioni:}}} L'attore è collegato al sistema e possiede una sessione attiva per un utente con ruolo ``gastroenterologist''.\\
\textsc{\textbf{Scenario principale:}}  L'attore invia una richiesta per ottenere l'esame desiderato, fornendone il codice identificativo interno .\\
\textsc{\textbf{Postcondizioni:}} L'attore ha ricevuto correttamente i dati richiesti al sistema.\\
\textsc{\textbf{Estensioni:}} L'attore invia una richiesta non valida (\hyperref[sec:UC16]{UC16}).

\section{UC21: Inserimento diagnosi per un esame}
\label{sec:UC21}
\textsc{\textbf{Attori:}} Front end.\\
\textsc{\textbf{Scopo e descrizione:}} L'attore vuole inserire una diagnosi per un esame.\\
\textsc{\textsc{\textbf{Precondizioni:}}} L'attore è collegato al sistema e possiede una sessione attiva per un utente con ruolo ``gastroenterologist'', l'esame rientra tra quelli visibili.\\
\textsc{\textbf{Scenario principale:}}  L'attore invia una richiesta di inserimento della diagnosi per l'esame desiderato contenente le informazioni generali e specifiche della diagnosi, fornendo il codice identificativo interno dell'esame.\\
\textsc{\textbf{Postcondizioni:}} Il sistema ha registrato correttamente la diagnosi e l'attore ha ricevuto una risposta che conferma il successo dell'operazione.\\
\textsc{\textbf{Estensioni:}} L'attore invia una richiesta non valida (\hyperref[sec:UC16]{UC16}).

\section{UC22: Inserimento della tabella della dieta}
\label{sec:UC22}
\textsc{\textbf{Attori:}} Front end.\\
\textsc{\textbf{Scopo e descrizione:}} L'attore vuole inserire la tabella della dieta associata ad un esame.\\
\textsc{\textsc{\textbf{Precondizioni:}}} L'attore è collegato al sistema e possiede una sessione attiva per un utente con ruolo ``gastroenterologist'', l'esame rientra tra quelli visibili.\\
\textsc{\textbf{Scenario principale:}}  L'attore invia una richiesta di inserimento della tabella della dieta per l'esame desiderato, fornendo il codice identificativo interno dell'esame.\\
\textsc{\textbf{Postcondizioni:}} Il sistema ha registrato correttamente la tabella della dieta e l'attore ha ricevuto una risposta che conferma il successo dell'operazione.\\
\textsc{\textbf{Estensioni:}} L'attore invia una richiesta non valida (\hyperref[sec:UC16]{UC16}).

\section{UCG4: Operazioni per utente ``bmr''}
\label{sec:UCG4}
Questo caso d'uso generale riassume le operazioni disponibili al \textit{front end} quando è attiva una sessione per un utente con ruolo di ``bmr'' (supervisore del laboratorio). Valgono le stesse considerazioni fatte per \hyperref[sec:UCG3]{UCG3}.
\textsc{\textbf{Attori:}} Front end.\\
\textsc{\textbf{Scopo e descrizione:}} L'attore vuole eseguire operazioni permesse ad un utente con ruolo di ``bmr''.\\
\textsc{\textbf{Precondizioni:}} L'attore è collegato al sistema e possiede una sessione attiva per un utente con ruolo ``bmr''.\\
\textsc{\textbf{Scenario principale:}} 
\begin{itemize}
    \item L'attore richiede la lista di esami per cui è stata emessa una diagnosi (\hyperref[sec:UC23]{UC23});
    \item L'attore richiede i dati di un esame per cui è stata emessa una diagnosi (\hyperref[sec:UC24]{UC24});
    \item L'attore aggiorna una diagnosi per un esame (\hyperref[sec:UC25]{UC25});
    \item L'attore aggiorna la tabella della dieta per un esame (\hyperref[sec:UC26]{UC26});
    \item L'attore effettua la disconnessione dalla piattaforma (\hyperref[sec:UC18]{UC18}).
\end{itemize}
\textsc{\textbf{Postcondizioni:}} L'attore ha correttamente portato a termine le operazioni desiderate.

\section{UC23: Richiesta lista esami con diagnosi}
\label{sec:UC23}
\textsc{\textbf{Attori:}} Front end.\\
\textsc{\textbf{Scopo e descrizione:}} L'attore vuole ottenere la lista degli esami per cui è stata emessa una diagnosi.\\
\textsc{\textsc{\textbf{Precondizioni:}}} L'attore è collegato al sistema e possiede una sessione attiva per un utente con ruolo ``bmr''.\\
\textsc{\textbf{Scenario principale:}} L'attore invia una richiesta per ottenere la lista di esami desiderata.\\
\textsc{\textbf{Postcondizioni:}} L'attore ha ricevuto correttamente i dati richiesti al sistema.\\
\textsc{\textbf{Estensioni:}} L'attore invia una richiesta non valida (\hyperref[sec:UC16]{UC16}).

\section{UC24: Richiesta informazioni esame per cui è stata emessa una diagnosi}
\label{sec:UC24}
\textsc{\textbf{Attori:}} Front end.\\
\textsc{\textbf{Scopo e descrizione:}} L'attore vuole ottenere i dati di un esame per cui è stata emessa una diagnosi.\\
\textsc{\textsc{\textbf{Precondizioni:}}} L'attore è collegato al sistema e possiede una sessione attiva per un utente con ruolo ``gastroenterologist''.\\
\textsc{\textbf{Scenario principale:}}  L'attore invia una richiesta per ottenere l'esame desiderato, fornendone il codice identificativo interno .\\
\textsc{\textbf{Postcondizioni:}} L'attore ha ricevuto correttamente i dati richiesti al sistema.\\
\textsc{\textbf{Estensioni:}} L'attore invia una richiesta non valida (\hyperref[sec:UC16]{UC16}).

\section{UC25: Aggiornamento diagnosi per un esame}
\label{sec:UC25}
\textsc{\textbf{Attori:}} Front end.\\
\textsc{\textbf{Scopo e descrizione:}} L'attore vuole modificare la diagnosi emessa per un esame.\\
\textsc{\textsc{\textbf{Precondizioni:}}} L'attore è collegato al sistema e possiede una sessione attiva per un utente con ruolo ``bmr'', l'esame rientra tra quelli visibili.\\
\textsc{\textbf{Scenario principale:}}  L'attore invia una richiesta di modifica della diagnosi per l'esame desiderato contenente le nuove informazioni generali e specifiche della diagnosi, fornendo il codice identificativo interno dell'esame.\\
\textsc{\textbf{Postcondizioni:}} Il sistema ha aggiornato correttamente la diagnosi e l'attore ha ricevuto una risposta che conferma il successo dell'operazione.\\
\textsc{\textbf{Estensioni:}} L'attore invia una richiesta non valida (\hyperref[sec:UC16]{UC16}).

\section{UC26: Aggiornamento tabella della dieta}
\label{sec:UC26}
\textsc{\textbf{Attori:}} Front end.\\
\textsc{\textbf{Scopo e descrizione:}} L'attore vuole aggiornare la tabella della dieta associata ad un esame.\\
\textsc{\textsc{\textbf{Precondizioni:}}} L'attore è collegato al sistema e possiede una sessione attiva per un utente con ruolo ``bmr'', l'esame rientra tra quelli visibili.\\
\textsc{\textbf{Scenario principale:}}  L'attore invia una richiesta di modifica della tabella della dieta per l'esame desiderato, fornendo il codice identificativo interno dell'esame.\\
\textsc{\textbf{Postcondizioni:}} Il sistema ha aggiornato correttamente la tabella della dieta e l'attore ha ricevuto una risposta che conferma il successo dell'operazione.\\
\textsc{\textbf{Estensioni:}} L'attore invia una richiesta non valida (\hyperref[sec:UC16]{UC16}).

\section{UCG5: Operazioni per il gestionale del laboratorio}
\label{sec:UCG5}
Questo caso d'uso generale riassume le operazioni disponibili al software gestionale del laboratorio attraverso gli appositi \textit{endpoint} \ref{itm:http}. Si prevede che le richieste \ref{itm:http} contengano, nella sezione \textit{header}, le credenziali necessarie per l'autenticazione tramite HMAC.\\ 
\textsc{\textbf{Attori:}} Gestionale laboratorio.\\
\textsc{\textbf{Scopo e descrizione:}} L'attore vuole eseguire operazioni rese disponibili dagli \textit{endpoint} esposti.\\
\textsc{\textbf{Precondizioni:}} L'attore ha accesso agli \textit{endpoint} forniti dal sistema e possiede le credenziali necessarie per l'autenticazione HMAC.\\
\textsc{\textbf{Scenario principale:}} 
\begin{itemize}
    \item L'attore richiede la lista di esami il cui campione è stato raccolto (\hyperref[sec:UC27]{UC27});
    \item L'attore aggiorna lo stato di avanzamento di un esame (\hyperref[sec:UC28]{UC28});
    \item L'attore rifiuta il campione per un esame (\hyperref[sec:UC29]{UC29});
    \item L'attore carica il referto delle analisi (\hyperref[sec:UC30]{UC30}).
\end{itemize}
\textsc{\textbf{Postcondizioni:}} L'attore ha correttamente portato a termine le operazioni desiderate.

\section{UC27: Richiesta lista esami con campione}
\label{sec:UC27}
\textsc{\textbf{Attori:}} Gestionale laboratorio.\\
\textsc{\textbf{Scopo e descrizione:}} L'attore vuole ottenere la lista degli esami il cui campione è stato raccolto, compresi quelli il cui iter di avanzamento è stato completato.\\
\textsc{\textsc{\textbf{Precondizioni:}}} L'attore ha accesso all'\textit{endpoint} corrispondente e ha autenticato correttamente la richiesta.\\
\textsc{\textbf{Scenario principale:}}  L'attore invia la richiesta \ref{itm:http} di sincronizzazione della lista degli esami all'\textit{endpoint} appropriato.\\
\textsc{\textbf{Postcondizioni:}} L'attore ha ricevuto una risposta contenente i dati richiesti in formato \ref{itm:json}.

\section{UC28: Aggiornamento stato di avanzamento di un esame}
\label{sec:UC28}
\textsc{\textbf{Attori:}} Gestionale laboratorio.\\
\textsc{\textbf{Scopo e descrizione:}} L'attore vuole aggiornare lo stato di avanzamento di un esame ad uno di quello disponibili.\\
\textsc{\textsc{\textbf{Precondizioni:}}} L'attore ha accesso all'\textit{endpoint} corrispondente e ha autenticato correttamente la richiesta.\\
\textsc{\textbf{Scenario principale:}}  L'attore invia la richiesta \ref{itm:http} corretta per il nuovo stato dell'esame all'\textit{endpoint} appropriato, specificando il codice campione a cui l'esame è associato.\\
\textsc{\textbf{Postcondizioni:}} L'attore ha ricevuto una risposta che conferma il successo dell'operazione.

\section{UC29: Rifiuto di un campione}
\label{sec:UC29}
\textsc{\textbf{Attori:}} Gestionale laboratorio.\\
\textsc{\textbf{Scopo e descrizione:}} L'attore vuole segnalare che il campione per un determinato esame è stato rifiutato.\\
\textsc{\textsc{\textbf{Precondizioni:}}} L'attore ha accesso all'\textit{endpoint} corrispondente e ha autenticato correttamente la richiesta.\\
\textsc{\textbf{Scenario principale:}}  
\begin{itemize}
    \item L'attore invia la richiesta \ref{itm:http} per rifiutare del campione all'\textit{endpoint} appropriato, specificando il codice campione a cui l'esame è associato e inserendo nel \textit{body} la ragione del rifiuto.
    \item Il sistema effettua il \textit{parsing} della risposta per ottenere la ragione del rifiuto e aggiorna l'esame desiderato.
\end{itemize}
\textsc{\textbf{Postcondizioni:}} Il sistema ha aggiornato correttamente l'esame e l'attore ha ricevuto una risposta che conferma il successo dell'operazione.

\section{UC30: Caricamento referto delle analisi}
\label{sec:UC30}
\textsc{\textbf{Attori:}} Gestionale laboratorio.\\
\textsc{\textbf{Scopo e descrizione:}} L'attore vuole caricare il referto delle analisi per un esame, assieme ad eventuali appunti sul referto.\\
\textsc{\textsc{\textbf{Precondizioni:}}} L'attore ha accesso all'\textit{endpoint} corrispondente, ha autenticato correttamente la richiesta ed è disponibile un referto da caricare.\\
\textsc{\textbf{Scenario principale:}}  \begin{itemize}
    \item L'attore invia la richiesta \ref{itm:http} di conclusione delle analisi all'\textit{endpoint} appropriato, allegando il referto ed eventuali note alle analisi e specificando il codice campione a cui l'esame è associato.
    \item Il sistema effettua il \textit{parsing} della risposta e aggiorna il record dell'esame corrispondente aggiungendo il referto ed eventuali note.
\end{itemize}
\textsc{\textbf{Postcondizioni:}} Il sistema ha ricevuto il referto, ha aggiornato correttamente l'esame e l'attore ha ricevuto una risposta che conferma il successo dell'operazione.

\setlength{\parindent}{12pt}

\section{Tracciamento dei requisiti}
TBD
\subsection{Notazione}
\subsection{Requisiti funzionali}
\subsection{Requisiti qualitativi}
\subsection{Requisiti di vincolo}