%!TEX root = ../dissertation.tex
\begin{savequote}[75mm]
  Era l'immagine di un compimento, e perciò qualcosa di assieme vitale e funebre.
\qauthor{Francesco Targhetta}
\end{savequote}
\chapter{Conclusione}
\label{conclusion}
Alla conclusione del progetto mi è stato possibile esprimere due tipi di valutazione sui risultati ottenuti: una oggettiva, basata su criteri misurabili quali il rispetto delle tempistiche e il soddisfacimento degli obiettivi, e una personale, data dall'impatto avuto dall'ambiente e dal metodo di lavoro nonché dalle conoscenze acquisite in corso di progetto. 

Volendo fare una considerazione generale, l'esperienza ha avuto un'esito positivo: gli obiettivi fissati sono stati raggiunti senza grosse difficoltà, il prodotto realizzato è stato giudicato pronto ad essere utilizzato ed è stata espressa soddisfazione sul risultato del lavoro compiuto. Allo stesso modo ho avuto occasione di apprezzare un ambiente di lavoro accogliente e di grande aiuto alla mia crescita personale e professionale; e per questo voglio ringraziare lo staff di Moku.
\section{Valutazione oggettiva}
\subsection{Raggiungimento degli obiettivi}
Alla conclusione del periodo di stage, è possibile affermare che tutti gli obiettivi obbligatori e desiderabili posti in fase di pianificazione sono stati raggiunti con pieno successo. Per quanto riguarda gli obiettivi facoltativi, uno di essi è stato sviluppato ma non è stato rilasciato entro la fine dello stage (requisito F01, \hyperref[sec:2.2.3]{sezione 2.2.3.1}, mentre ne sono stati implementati altri che portavano valore aggiunto a funzionalità già presenti.

Di seguito viene riportata una tabella che riassume gli obiettivi pianificati in riferimento alla \hyperref[sec:2.2.3]{sezione 2.2.3.1} ed il loro stato di raggiungimento al termine del lavoro:
\begin{table}[h]
    \begin{center}
        \begin{tabular}{|c||c|}
          \hline % linea orizzontale
          \hspace{5pt}\textbf{Requisito}\hspace{5pt} & \textbf{Stato}  \\\hline\hline
          \textbf{O01} & Soddisfatto \cr\hline
          \textbf{O01} & Soddisfatto \cr\hline
          \textbf{O03}  & Soddisfatto \cr\hline\hline
          \textbf{D01} &  Soddisfatto \cr\hline
          \textbf{D02} &  Soddisfatto \cr\hline
          \textbf{D03} &  Soddisfatto \cr\hline
          \textbf{D04} &  Soddisfatto \cr\hline\hline
          \textbf{F01} &  Soddisfatto, non rilasciato \cr\hline
          \textbf{F02} &  Soddisfatto \cr\hline
        \end{tabular}
        \caption{Soddisfacimento degli obiettivi pianificati.}
        \label{tab:finreq}
    \end{center}
    \end{table}
\vspace{-25pt}
\subsection{Consuntivo}
Per quanto riguarda la pianificazione temporale, il limite di 300 ore è stato rispettato ed è risultato sufficiente, come esposto, per raggiungere gli obiettivi posti. Tuttavia, a causa di impegni accademici e alcuni giorni malattia, nom è bastata la settimana di \textit{slack} prevista e il lavoro è stato distribuito anche nella settimana successiva, previa richiesta all'ufficio di Ateneo responsabile; le attività sono dunque giunte al termine il 17 luglio, contrariamente alla data prevista del 12 luglio (usando la settimana di \textit{slack}). Ciò è evidentemente dovuto da una stima errata del tempo da dedicare agli impegni accademici concorrenti al periodo di stage, e l'eventuale dilazione dei termini era prevedibile. 

Rispetto alla suddivisione oraria pianificata, sono state effettuate alcune modifiche in corso d'opera. Alcune attività, quali lo studio di nuove tecnologie e l'analisi dei requisiti, sono state ``spalmate'' nel periodo di stage in quanto necessarie dopo ogni \textit{release} del prodotto per rivalutare le esigenze in collaborazione con il cliente. la preparazione dell'ambiente di sviluppo e la produzione di documentazione hanno richiesto meno tempo rispetto a quanto previsto grazie alle \textit{best practices} in vigore nell'azienda, che ne hanno accelerati i tempi; le ore così risparmiate sono state assegnate alla progettazione e all'implementazione di test. La progettazione in particolare ha più che raddoppiato le ore previste, date le modifiche ai requisiti avvenute nelle prime settimane del progetto e la necessità di acquisire confidenza con la struttura dell'applicazione Rails. in compenso, una progettazione più dettagliata e ragionata ha permesso di risparmiare tempo in fase implementativa, permettendo di codificare le funzionalità previste senza aggiunta di errori (ottenendo quindi correttezza per costruzione).

La tabella seguente riassume la ridistribuzione oraria al termine dello stage:

\begin{table}[h]
    \begin{center}
        \begin{tabular}{|c|c|}
          \hline % linea orizzontale
          \hspace{5pt}\textbf{Attività}\hspace{5pt} & \textbf{Monte ore}  \\\hline
          Comprensione sistema e obiettivi & 20 \cr\hline
          Analisi dei requisiti & 40 \cr\hline
          Progettazione  & 48 (+28) \cr\hline
          Studio e setup ambiente di sviluppo &  8 (-12) \cr\hline
          Implementazione &  134 (-16) \cr\hline
          Test e validazione &  40 (+10) \cr\hline
          Documentazione &  10 (-10) \cr\hline\hline
          \textbf{Totale} &  300 \cr\hline
        \end{tabular}
        \caption{Totale di ore dedicato a ciascuna attività al termine dello stage.}
        \label{tab:orefine}
    \end{center}
    \end{table}

\section{Valutazione personale}
\subsection{Conoscenze acquisite}
L'attività di stage volta mi ha permesso di apprendere e raffinare molte conoscenze, sia tecniche che professionali. Dal punto di vista tecnico, lavorare con un linguaggio di programmazione a me sconosciuto e con tecnologie molto differenti da quelle utilizzate in ambito accademico ha rappresentato una sfida molto interessante, che mi ha permesso di espandere i miei orizzonti in termini di paradigmi sia architetturali che implementativi, senza contare il valore aggiunto portato dall'avere appreso come sviluppare un'applicazione in Rails, un \textit{framework} con grosse potenzialità. Ho inoltre imparato ad usare in modo più vantaggioso gli strumenti di sviluppo, in particolare le funzionalità di \textit{debugging} di RubyMine, e di supporto: l'introduzione al paradigma \textit{git-flow} è stata illuminante e lo utilizzerò sicuramente in progetti futuri.

L'interazione con il tutor aziendale, ma anche il resto del personale di Moku, è stata estremamente utile per acquisire esperienza sulle responsabilità e dui compiti di uno sviluppatore, nonché su come rapportarsi ad un cliente. Dalla capacità di affrontare difficoltà progettuali all'abilità di selezionare la tecnologia e l'approccio ottimali per soddisfare un requisito, sono molto soddisfatto delle conoscenze professionali acquisite durante questo progetto.

\subsection{Crescita personale}
Al termine dell'esperienza di stage, sono arrivato a considerarla com un importante passo anche nella mia crescita personale. Il contatto con l'ambiente lavorativo, le responsabilità assunte nel creare un prodotto che verrà utilizzato in un contesto reale e le sfide di \textit{problem solving} che mi si sono presentate mi hanno permesso di ampliare i miei orizzonti e migliorarmi sia come persona che come futuro professionista. Moku mi ha accolto con un ambiente giovane, cordiale ed accogliente, in cui è stato possibile comunicare chiaramente e in cui le mie esigenze da studente sono state comprese. In conclusione, sono molto felice dell'esperienza fatta e la ritengo un passo molto importante della mia educazione.