%!TEX root = ../dissertation.tex
\externaldocument{./chapter2}
\appendix
\chapter{Appendice A}
\label{sec:AppendixA}
\section{Casi d'uso secondari}
Di seguito vengono riportati i casi d'uso secondari per ogni caso d'uso principale, in riferimento alla sezione \hyperref[sec:usecase]{3.2} del \hyperref[sec:chap3]{capitolo 3}.
\setlength{\parindent}{0cm}
\subsection{Casi d'uso secondari per UC1}
\subsubsection{UC1.1: Inserimento email}
\label{sec:uc11}
\textsc{\textbf{Attori:}} Utente non autenticato.\\
\textsc{\textbf{Scopo e descrizione:}} L'attore vuole inserire un indirizzo email per l'autenticazione al pannello di amministrazione.\\
\textsc{\textsc{\textbf{Precondizioni:}}} L'attore ha accesso alla pagina di autenticazione.\\
\textsc{\textbf{Scenario principale:}} L'attore inserisce un indirizzo mail nell'apposito campo.\\
\textsc{\textbf{Postcondizioni:}} L'attore ha inserito un indirizzo email valido.

\subsubsection{UC1.2: Inserimento password}
\label{sec:uc12}
\textsc{\textbf{Attori:}} Utente non autenticato.\\
\textsc{\textbf{Scopo e descrizione:}} L'attore vuole inserire una password per l'autenticazione al pannello di amministrazione.\\
\textsc{\textsc{\textbf{Precondizioni:}}} L'attore ha accesso alla pagina di autenticazione.\\
\textsc{\textbf{Scenario principale:}} L'attore inserisce una password nel campo apposito.\\
\textsc{\textbf{Postcondizioni:}} L'attore ha inserito una password corretta.

\subsection{Casi d'uso secondari per UC6}
\subsubsection{UC6.1: Modifica ordine della tabella dei record}
\label{sec:UC61}
\textsc{\textbf{Attori:}} Amministratore.\\
\textsc{\textbf{Scopo e descrizione:}} L'attore vuole cambiare l'ordine della tabella dei record secondo un certo campo.\\
\textsc{\textsc{\textbf{Precondizioni:}}} L'attore sta visualizzando una tabella di record.\\
\textsc{\textbf{Scenario principale:}} L'attore seleziona la colonna del campo in base al quale ordinare la tabella.
\textsc{\textbf{Postcondizioni:}} L'attore ha riordinato la tabella secondo il campo desiderato.

\subsubsection{UC6.2: Applicazione filtro alla tabella dei record}
\label{sec:UC62}
\textsc{\textbf{Attori:}} Amministratore.\\
\textsc{\textbf{Scopo e descrizione:}} L'attore vuole filtrare alcune righe della tabella dei record.\\
\textsc{\textsc{\textbf{Precondizioni:}}} L'attore sta visualizzando una tabella di record.\\
\textsc{\textbf{Scenario principale:}} 
\begin{itemize}
    \item L'attore seleziona l'icona del filtro;
    \item L'attore imposta il filtro desiderato;
    \item L'attore clicca sul pulsante``Filtra".
\end{itemize}
\textsc{\textbf{Postcondizioni:}} L'attore ha applicato il filtro desiderato alla tabella dei record.

\subsubsection{UC6.3: Rimozione filtro alla tabella dei record}
\label{sec:UC63}
\textsc{\textbf{Attori:}} Amministratore.\\
\textsc{\textbf{Scopo e descrizione:}} L'attore vuole rimuovere un filtro dalla tabella dei record.\\
\textsc{\textsc{\textbf{Precondizioni:}}} L'attore sta visualizzando una tabella di record con un filtro attivo.\\
\textsc{\textbf{Scenario principale:}} 
\begin{itemize}
    \item L'attore seleziona l'icona del filtro;
    \item L'attore clicca sul pulsante ``Rimuovi filtri".
\end{itemize}
\textsc{\textbf{Postcondizioni:}} L'attore ha rimosso il filtro desiderato alla tabella dei record.

\subsubsection{UC6.4: Aggiunta di un nuovo record}
\label{sec:UC64}
\textsc{\textbf{Attori:}} Amministratore.\\
\textsc{\textbf{Scopo e descrizione:}} L'attore vuole aggiungere un record alla tabella.\\
\textsc{\textsc{\textbf{Precondizioni:}}} L'attore sta visualizzando una tabella di record.\\
\textsc{\textbf{Scenario principale:}} 
\begin{itemize}
    \item L'attore clicca sul pulsante ``Aggiungi \textbf{\{record\}}", dove \textbf{\{record\}} è il nome della categoria selezionata;
    \item L'attore viene reindirizzato al \textit{form} di creazione del record;
    \item L'attore riempie gli appositi campi con i dati desiderati;
    \item L'attore clicca sul pulsante ``Crea \textbf{\{record\}}".
\end{itemize}
\textsc{\textbf{Postcondizioni:}} L'attore ha creato con successo un nuovo record.\\
\textsc{\textbf{Estensioni:}} L'attore annulla la creazione di un nuovo record (\hyperref[sec:UC69]{UC6.9}).

\subsubsection{UC6.5: Visualizzazione di un record}
\label{sec:UC65}
\textsc{\textbf{Attori:}} Amministratore.\\
\textsc{\textbf{Scopo e descrizione:}} L'attore vuole visualizzare un record della tabella.\\
\textsc{\textsc{\textbf{Precondizioni:}}} L'attore sta visualizzando una tabella do record.\\
\textsc{\textbf{Scenario principale:}} 
\begin{itemize}
    \item L'attore clicca sul pulsante ``Mostra" sulla riga corrispondente al record desiderato;
    \item L'attore visualizza i dati relativi al record;
    \item L'attore modifica il record;
    \item L'attore elimina il record.
\end{itemize}
\textsc{\textbf{Postcondizioni:}} L'attore ha eseguito tutte le operazioni di gestione della categoria di record selezionata con successo.

\subsubsection{UC6.6: Modifica di un record}
\label{sec:UC66}
\textsc{\textbf{Attori:}} Amministratore.\\
\textsc{\textbf{Scopo e descrizione:}} L'attore vuole modificare un record della tabella.\\
\textsc{\textsc{\textbf{Precondizioni:}}} L'attore sta visualizzando una tabella di record.\\
\textsc{\textbf{Scenario principale:}}
\begin{itemize}
    \item L'attore clicca sul pulsante ``Modifica'' sulla riga corrispondente al record desiderato;
    \item L'attore viene reindirizzato al \textit{form} di modifica del record;
    \item L'attore modifica il contenuto dei campi disponibili;
    \item L'attore clicca sul pulsante ``Aggiorna \textbf{\{record\}}", dove \textbf{\{record\}} è il nome della categoria selezionata.
\end{itemize}
\textsc{\textbf{Scenario alternativo:}}
\begin{itemize}
    \item L'attore visualizza il record desiderato (\hyperref[sec:UC65]{UC6.5});
    \item L'attore clicca sul pulsante ``Modifica \textbf{\{record\}}", dove \textbf{\{record\}} è il nome della categoria selezionata;
    \item L'attore viene reindirizzato al \textit{form} di modifica del record;
    \item L'attore modifica il contenuto dei campi disponibili;
    \item L'attore clicca sul pulsante ``Aggiorna \textbf{\{record\}}";
\end{itemize}
\textsc{\textbf{Postcondizioni:}} L'attore ha correttamente aggiornato il record con i dati desiderati.
\textsc{\textbf{Estensioni:}} \begin{itemize}
    \item L'attore annulla la modifica del record (\hyperref[sec:UC69]{UC6.9});
    \item L'attore commette un errore nella modifica del record (\hyperref[sec:UC69]{UC6.9}).
\end{itemize}

\subsubsection{UC6.7: Eliminazione di un record}
\label{sec:UC67}
\textsc{\textbf{Attori:}} Amministratore.\\
\textsc{\textbf{Scopo e descrizione:}} L'attore vuole eliminare un record della tabella.\\
\textsc{\textsc{\textbf{Precondizioni:}}} L'attore sta visualizzando una tabella di record.\\
\textsc{\textbf{Scenario principale:}}
\begin{itemize}
    \item L'attore clicca sul pulsante ``Rimuovi'' sulla riga corrispondente al record desiderato;
    \item L'attore visualizza un \textit{pop-up} di conferma dell'operazione;
    \item L'attore clicca sul pulsante ``Ok'' per confermare l'operazione.
\end{itemize}
\textsc{\textbf{Scenario alternativo:}}
\begin{itemize}
    \item L'attore visualizza il record desiderato (\hyperref[sec:UC65]{UC6.5});
    \item L'attore clicca sul pulsante ``Rimuovi \textbf{\{record\}}", dove \textbf{\{record\}} è il nome della categoria selezionata;
    \item L'attore visualizza un \textit{pop-up} di conferma dell'operazione;
    \item L'attore clicca sul pulsante ``Ok'' per confermare l'operazione.
\end{itemize}
\textsc{\textbf{Postcondizioni:}} L'attore ha correttamente rimosso il record.
\textsc{\textbf{Estensioni:}} L'attore annulla la rimozione del record (\hyperref[sec:UC69]{UC6.9}).

\subsubsection{UC6.8: Errore nella modifica di un record}
\label{sec:UC68}
\textsc{\textbf{Attori:}} Amministratore.\\
\textsc{\textbf{Scopo e descrizione:}} L'attore vuole modificare un record.\\
\textsc{\textsc{\textbf{Precondizioni:}}} L'attore sta visualizzando il \textit{form} di modifica del record.\\
\textsc{\textbf{Scenario principale:}} 
\begin{itemize}
    \item L'attore inserisce uno o più valori non validi in uno o più dei campi disponibili;
    \item L'attore visualizza un messaggio di errore, il quale descrive il valore invalido inserito e l'errore riscontrato.
\end{itemize}
\textsc{\textbf{Postcondizioni:}} Il messaggio di errore rimane in cima al \textit{form} di modifica e la modifica viene prevenuta.

\subsubsection{UC6.9: Annullamento operazione}
\label{sec:UC69}
\textsc{\textbf{Attori:}} Amministratore.\\
\textsc{\textbf{Scopo e descrizione:}} L'attore vuole annullare l'operazione in corso.\\
\textsc{\textsc{\textbf{Precondizioni:}}} L'attore non ha ancora terminato l'operazione precedentemente selezionata.\\
\textsc{\textbf{Scenario principale:}} L'attore clicca sul pulsante ``Annulla".\\
\textsc{\textbf{Postcondizioni:}} L'attore ha annullato correttamente l'operazione ed eventuali modifiche vengono prevenute.

\subsubsection{UC6.10: Gestione di un utente}
\label{sec:UC610}
\textsc{\textbf{Attori:}} Amministratore.\\
\textsc{\textbf{Scopo e descrizione:}} L'attore vuole gestire un utente autorizzato.\\
\textsc{\textsc{\textbf{Precondizioni:}}} L'attore ha seleziona la scheda ``Utenti''.\\
\textsc{\textbf{Scenario principale:}} 
\begin{itemize}
    \item L'attore visualizza il record corrispondente all'utente desiderato (\hyperref[sec:UC65]{UC6.5});
    \item L'attore genera una nuova password per l'utente (\hyperref[sec:UC6101]{UC6.10.1}).
\end{itemize}
\textsc{\textbf{Postcondizioni:}} L'attore ha eseguito tutte le operazioni di gestione dell'utente desiderato con successo.

\subsubsection{UC6.10.1: Generazione nuova password per un utente}
\label{sec:UC6101}
\textsc{\textbf{Attori:}} Amministratore.\\
\textsc{\textbf{Scopo e descrizione:}} L'attore vuole generare una nuova password per un utente autorizzato.\\
\textsc{\textsc{\textbf{Precondizioni:}}} L'attore sta visualizzando il record relativo all'utente.\\
\textsc{\textbf{Scenario principale:}} L'attore preme sul pulsante ``Rigenera password''.\\
\textsc{\textbf{Postcondizioni:}} L'attore ha correttamente generato una nuova password per l'utente desiderato.

\subsubsection{UC6.11: Gestione di un cliente}
\label{sec:UC611}
\textsc{\textbf{Attori:}} Amministratore.\\
\textsc{\textbf{Scopo e descrizione:}} L'attore vuole gestire un cliente.\\
\textsc{\textsc{\textbf{Precondizioni:}}} L'attore ha seleziona la scheda ``Clienti''.\\
\textsc{\textbf{Scenario principale:}} 
\begin{itemize}
    \item L'attore visualizza il record corrispondente all'utente desiderato (\hyperref[sec:UC65]{UC6.5});
    \item L'attore scarica il modulo di consenso al trattamento dei dati (\hyperref[sec:UC6101]{UC6.11.1}).
\end{itemize}
\textsc{\textbf{Postcondizioni:}} L'attore ha eseguito tutte le operazioni di gestione del cliente desiderato con successo.

\subsubsection{UC6.11.1: Scaricamento modulo di consenso al trattamento dei dati}
\label{sec:UC6111}
\textsc{\textbf{Attori:}} Amministratore.\\
\textsc{\textbf{Scopo e descrizione:}} L'attore vuole scaricare il modulo per il consenso al trattamento dei dati firmato da un cliente (se presente).\\
\textsc{\textsc{\textbf{Precondizioni:}}} L'attore sta visualizzando il record relativo al cliente desiderato.\\
\textsc{\textbf{Scenario principale:}} L'attore preme sul pulsante ``Scarica privacy policy''.\\
\textsc{\textbf{Postcondizioni:}} L'attore ha correttamente scaricato il modulo (se presente).

\subsection{Casi d'uso secondari per UC11}
\subsubsection{UC11.1: Richiesta informazioni lista degli utenti}
\label{sec:UC111}
\textsc{\textbf{Attori:}} Front end.\\
\textsc{\textbf{Scopo e descrizione:}} L'attore vuole ottenere la lista degli utenti con ruolo ``pharmacy'' per la farmacia corrente.\\
\textsc{\textsc{\textbf{Precondizioni:}}} L'attore è collegato al sistema e possiede una sessione attiva per un utente con ruolo ``pharmacy''.\\
\textsc{\textbf{Scenario principale:}} L'attore richiede al sistema i dati degli utenti appartenenti alla farmacia corrente.\\
\textsc{\textbf{Postcondizioni:}} L'attore ha ricevuto correttamente i dati richiesti al sistema.\\
\textsc{\textbf{Estensioni:}} L'attore invia una richiesta non valida (\hyperref[sec:UC16]{UC16}).

\subsubsection{UC11.2: Richiesta informazioni utente}
\label{sec:UC112}
\textsc{\textbf{Attori:}} Front end.\\
\textsc{\textbf{Scopo e descrizione:}} L'attore vuole ottenere i dati di un utente con ruolo ``pharmacy'' per la farmacia corrente.\\
\textsc{\textsc{\textbf{Precondizioni:}}} L'attore è collegato al sistema e possiede una sessione attiva per un utente con ruolo ``pharmacy''.\\
\textsc{\textbf{Scenario principale:}} L'attore richiede al sistema i dati dell'utente desiderato fornendone il codice identificativo interno.\\
\textsc{\textbf{Postcondizioni:}} L'attore ha ricevuto correttamente i dati richiesti al sistema.\\
\textsc{\textbf{Estensioni:}} L'attore invia una richiesta non valida (\hyperref[sec:UC16]{UC16}).

\subsubsection{UC11.3: Richiesta reset della password di un utente}
\label{sec:UC113}
\textsc{\textbf{Attori:}} Front end.\\
\textsc{\textbf{Scopo e descrizione:}} L'attore vuole richiedere il reset della password per un utente con ruolo ``pharmacy''.\\
\textsc{\textsc{\textbf{Precondizioni:}}} L'attore è collegato al sistema e possiede una sessione attiva per un utente con ruolo ``pharmacy''.\\
\textsc{\textbf{Scenario principale:}} 
\begin{itemize}
    \item L'attore richiede al sistema il reset della password dell'utente desiderato fornendone il codice identificativo interno.
    \item Il sistema invia un'email per la procedura di reset all'indirizzo mail dell'utente specificato.
\end{itemize}
\textsc{\textbf{Postcondizioni:}} La richiesta di reset viene processata correttamente dal sistema.\\
\textsc{\textbf{Estensioni:}} L'attore invia una richiesta non valida (\hyperref[sec:UC16]{UC16}).

\subsubsection{UC11.4: Impostazione nuova password per un utente}
\label{sec:UC114}
\textsc{\textbf{Attori:}} Front end.\\
\textsc{\textbf{Scopo e descrizione:}} L'attore vuole impostare.\\
\textsc{\textsc{\textbf{Precondizioni:}}} L'attore è collegato al sistema, possiede una sessione attiva per un utente con ruolo ``pharmacy'' e possiede un \textit{token} valido per la reimpostazione della password per l'utente autorizzato desiderato.\\
\textsc{\textbf{Scenario principale:}} L'attore invia al sistema la richiesta di reimpostazione contenente il \textit{token} di reset e la nuova password desiderata.\\
\textsc{\textbf{Postcondizioni:}} Il sistema ha reimpostato correttamente la password per l'utente desiderato e l'attore ha ricevuto un messaggio di conferma del successo dell'operazione.\\
\textsc{\textbf{Estensioni:}} L'attore invia una richiesta non valida (\hyperref[sec:UC16]{UC16}).

\subsubsection{UC11.5: Creazione utente}
\label{sec:UC115}
\textsc{\textbf{Attori:}} Front end.\\
\textsc{\textbf{Scopo e descrizione:}} L'attore vuole creare un utente con ruolo ``pharmacy'' per la farmacia corrente.\\
\textsc{\textsc{\textbf{Precondizioni:}}} L'attore è collegato al sistema e possiede una sessione attiva per un utente con ruolo ``pharmacy''.\\
\textsc{\textbf{Scenario principale:}} L'attore invia al sistema una richiesta di creazione di un utente contenente un indirizzo email valido e una password.\\
\textsc{\textbf{Postcondizioni:}} Il sistema ha creato correttamente l'utente desiderato e l'attore ha ricevuto un messaggio di conferma del successo dell'operazione.\\
\textsc{\textbf{Estensioni:}} L'attore invia una richiesta non valida (\hyperref[sec:UC16]{UC16}).

\subsubsection{UC11.6: Rimozione utente}
\label{sec:UC116}
\textsc{\textbf{Attori:}} Front end.\\
\textsc{\textbf{Scopo e descrizione:}} L'attore vuole eliminare un utente con ruolo ``pharmacy'' per la farmacia corrente.\\
\textsc{\textsc{\textbf{Precondizioni:}}} L'attore è collegato al sistema e possiede una sessione attiva per un utente con ruolo ``pharmacy''.\\
\textsc{\textbf{Scenario principale:}} L'attore richiede al sistema di eliminare l'utente desiderato fornendone il codice identificativo interno.\\
\textsc{\textbf{Postcondizioni:}} Il sistema ha rimosso correttamente l'utente desiderato e l'attore ha ricevuto un messaggio di conferma del successo dell'operazione.\\
\textsc{\textbf{Estensioni:}} L'attore invia una richiesta non valida (\hyperref[sec:UC16]{UC16}).

\subsubsection{UC11.7: Aggiornamento dati utente}
\label{sec:UC117}
\textsc{\textbf{Attori:}} Front end.\\
\textsc{\textbf{Scopo e descrizione:}} L'attore vuole aggiornare i dati di un utente con ruolo ``pharmacy'' per la farmacia corrente.\\
\textsc{\textsc{\textbf{Precondizioni:}}} L'attore è collegato al sistema e possiede una sessione attiva per un utente con ruolo ``pharmacy''.\\
\textsc{\textbf{Scenario principale:}} L'attore invia al sistema una richiesta di aggiornamento dei dati dell'utente desiderato fornendone il codice identificativo interno e i nuovi dati da inserire.\\
\textsc{\textbf{Postcondizioni:}} Il sistema ha aggiornato correttamente l'utente desiderato e l'attore ha ricevuto un messaggio di conferma del successo dell'operazione.\\
\textsc{\textbf{Estensioni:}} L'attore invia una richiesta non valida (\hyperref[sec:UC16]{UC16}).

\subsection{Casi d'uso secondari per UC12}

\subsubsection{UC12.1: Richiesta storico degli esami}
\label{sec:UC121}
\textsc{\textbf{Attori:}} Front end.\\
\textsc{\textbf{Scopo e descrizione:}} L'attore vuole ottenere lo storico degli esami per la farmacia corrente.\\
\textsc{\textsc{\textbf{Precondizioni:}}} L'attore è collegato al sistema e possiede una sessione attiva per un utente con ruolo ``pharmacy''.\\
\textsc{\textbf{Scenario principale:}} L'attore richiede al sistema i dati degli esami effettuati dalla farmacia corrente.\\
\textsc{\textbf{Postcondizioni:}} L'attore ha ricevuto correttamente i dati richiesti al sistema.\\
\textsc{\textbf{Estensioni:}} L'attore invia una richiesta non valida (\hyperref[sec:UC16]{UC16}).

\subsubsection{UC12.2: Richiesta storico degli esami in base allo stato di avanzamento}
\label{sec:UC122}
\textsc{\textbf{Attori:}} Front end.\\
\textsc{\textbf{Scopo e descrizione:}} L'attore vuole ottenere lo storico degli esami per la farmacia corrente che si trovino in un certo stato di avanzamento.\\
\textsc{\textsc{\textbf{Precondizioni:}}} L'attore è collegato al sistema e possiede una sessione attiva per un utente con ruolo ``pharmacy''.\\
\textsc{\textsc{\textbf{Scenario principale:}}} 
\begin{itemize}
    \item L'attore richiede al sistema i dati dell'esame desiderato fornendone lo stato di avanzamento desiderato;
    \item Il sistema filtra gli esami in base allo stato specificato e invia una risposta contenente i dati relativi agli esami filtrati.
\end{itemize} 
\textsc{\textbf{Postcondizioni:}} L'attore ha ricevuto correttamente i dati richiesti al sistema.\\
\textsc{\textbf{Estensioni:}} L'attore invia una richiesta non valida (\hyperref[sec:UC16]{UC16}).

\subsubsection{UC12.3: Richiesta informazioni esame}
\label{sec:UC123}
\textsc{\textbf{Attori:}} Front end.\\
\textsc{\textbf{Scopo e descrizione:}} L'attore vuole ottenere i dati di un esame  effettuato presso la farmacia corrente.\\
\textsc{\textsc{\textbf{Precondizioni:}}} L'attore è collegato al sistema e possiede una sessione attiva per un utente con ruolo ``pharmacy''.\\
\textsc{\textbf{Scenario principale:}} L'attore richiede al sistema i dati dell'esame desiderato fornendone il codice identificativo interno.\\
\textsc{\textbf{Postcondizioni:}} L'attore ha ricevuto correttamente i dati richiesti al sistema.\\
\textsc{\textbf{Estensioni:}} L'attore invia una richiesta non valida (\hyperref[sec:UC16]{UC16}).

\subsubsection{UC12.4: Richiesta informazioni esame in base al campione}
\label{sec:UC124}
\textsc{\textbf{Attori:}} Front end.\\
\textsc{\textbf{Scopo e descrizione:}} L'attore vuole ottenere i dati di un esame  effettuato presso la farmacia corrente con uno specifico codice campione.\\
\textsc{\textsc{\textbf{Precondizioni:}}} L'attore è collegato al sistema e possiede una sessione attiva per un utente con ruolo ``pharmacy''.\\
\textsc{\textbf{Scenario principale:}}
\begin{itemize}
    \item L'attore richiede al sistema i dati dell'esame desiderato fornendone il codice identificativo del campione associato;
    \item Il sistema ricerca l'esame desiderato e invia una risposta contenente i relativi dati.
\end{itemize} 
\textsc{\textbf{Postcondizioni:}} L'attore ha ricevuto correttamente i dati richiesti al sistema.\\
\textsc{\textbf{Estensioni:}} L'attore invia una richiesta non valida (\hyperref[sec:UC16]{UC16}).

\subsubsection{UC12.5: Creazione esame}
\label{sec:UC125}
\textsc{\textbf{Attori:}} Front end.\\
\textsc{\textbf{Scopo e descrizione:}} L'attore vuole creare un esame presso la farmacia corrente.\\
\textsc{\textsc{\textbf{Precondizioni:}}} L'attore è collegato al sistema e possiede una sessione attiva per un utente con ruolo ``pharmacy''.\\
\textsc{\textbf{Scenario principale:}} L'attore invia al sistema una richiesta di creazione di un esame contenente il codice identificativo interno del cliente associato e il codice del campione assegnato.\\
\textsc{\textbf{Postcondizioni:}} Il sistema ah creato correttamente l'esame e l'attore ha ricevuto un messaggio di conferma del successo dell'operazione.\\
\textsc{\textbf{Estensioni:}} L'attore invia una richiesta non valida (\hyperref[sec:UC16]{UC16}).

\subsubsection{UC12.6: Rimozione esame}
\label{sec:UC126}
\textsc{\textbf{Attori:}} Front end.\\
\textsc{\textbf{Scopo e descrizione:}} L'attore vuole eliminare un esame creato presso la farmacia corrente.\\
\textsc{\textsc{\textbf{Precondizioni:}}} L'attore è collegato al sistema e possiede una sessione attiva per un utente con ruolo ``pharmacy''.\\
\textsc{\textbf{Scenario principale:}} L'attore richiede al sistema di eliminare l'esame desiderato fornendone il codice identificativo interno.\\
\textsc{\textbf{Postcondizioni:}} Il sistema ha rimosso correttamente l'esame desiderato e l'attore ha ricevuto un messaggio di conferma del successo dell'operazione.\\
\textsc{\textbf{Estensioni:}} L'attore invia una richiesta non valida (\hyperref[sec:UC16]{UC16}).

\subsubsection{UC12.7: Aggiornamento dati esame}
\label{sec:UC127}
\textsc{\textbf{Attori:}} Front end.\\
\textsc{\textbf{Scopo e descrizione:}} L'attore vuole aggiornare i dati di un esame effettuato presso la farmacia corrente.\\
\textsc{\textsc{\textbf{Precondizioni:}}} L'attore è collegato al sistema e possiede una sessione attiva per un utente con ruolo ``pharmacy''.\\
\textsc{\textbf{Scenario principale:}} L'attore invia al sistema una richiesta di aggiornamento dei dati dell'esame desiderato fornendone il codice identificativo interno e i nuovi dati da inserire.\\
\textsc{\textbf{Postcondizioni:}} Il sistema ha aggiornato correttamente l'esame desiderato e l'attore ha ricevuto un messaggio di conferma del successo dell'operazione.\\
\textsc{\textbf{Estensioni:}} L'attore invia una richiesta non valida (\hyperref[sec:UC16]{UC16}).

\subsubsection{UC12.8: Aggiornamento risposte al questionario di un esame}
\label{sec:UC128}
\textsc{\textbf{Attori:}} Front end.\\
\textsc{\textbf{Scopo e descrizione:}} L'attore vuole aggiornare le risposte date al questionario di un esame.\\
\textsc{\textsc{\textbf{Precondizioni:}}} L'attore è collegato al sistema e possiede una sessione attiva per un utente con ruolo ``pharmacy''.\\
\textsc{\textbf{Scenario principale:}} L'attore invia al sistema una richiesta di aggiornamento delle risposte al questionario dell'esame desiderato fornendone il codice identificativo interno e le risposte da inserire.\\
\textsc{\textbf{Postcondizioni:}} Il sistema ha aggiornato correttamente le risposte al questionario associato all'esame desiderato e l'attore ha ricevuto un messaggio di conferma del successo dell'operazione.\\
\textsc{\textbf{Estensioni:}} L'attore invia una richiesta non valida (\hyperref[sec:UC16]{UC16}).

\subsection{Casi d'uso secondari per UC13}
\subsubsection{UC13.1: Richiesta informazioni registro dei clienti}
\label{sec:UC131}
\textsc{\textbf{Attori:}} Front end.\\
\textsc{\textbf{Scopo e descrizione:}} L'attore vuole ottenere il registro dei clienti della farmacia corrente.\\
\textsc{\textsc{\textbf{Precondizioni:}}} L'attore è collegato al sistema e possiede una sessione attiva per un utente con ruolo ``pharmacy''.\\
\textsc{\textbf{Scenario principale:}} L'attore richiede al sistema i dati dei clienti appartenenti alla farmacia corrente.\\
\textsc{\textbf{Postcondizioni:}} L'attore ha ricevuto correttamente i dati richiesti al sistema.\\
\textsc{\textbf{Estensioni:}} L'attore invia una richiesta non valida (\hyperref[sec:UC16]{UC16}).

\subsubsection{UC13.2: Richiesta informazioni cliente}
\label{sec:UC132}
\textsc{\textbf{Attori:}} Front end.\\
\textsc{\textbf{Scopo e descrizione:}} L'attore vuole ottenere i dati di un cliente della farmacia corrente.\\
\textsc{\textsc{\textbf{Precondizioni:}}} L'attore è collegato al sistema e possiede una sessione attiva per un utente con ruolo ``pharmacy''.\\
\textsc{\textbf{Scenario principale:}} L'attore richiede al sistema i dati di un cliente appartenente alla farmacia corrente fornendone il codice identificativo interno.\\
\textsc{\textbf{Postcondizioni:}} L'attore ha ricevuto correttamente i dati richiesti al sistema.\\
\textsc{\textbf{Estensioni:}} L'attore invia una richiesta non valida (\hyperref[sec:UC16]{UC16}).

\subsubsection{UC13.3: Creazione cliente}
\label{sec:UC133}
\textsc{\textbf{Attori:}} Front end.\\
\textsc{\textbf{Scopo e descrizione:}} L'attore vuole creare un cliente per la farmacia corrente.\\
\textsc{\textsc{\textbf{Precondizioni:}}} L'attore è collegato al sistema e possiede una sessione attiva per un utente con ruolo ``pharmacy''.\\
\textsc{\textbf{Scenario principale:}} L'attore invia al sistema una richiesta di creazione di un cliente contenente i dati necessari.\\
\textsc{\textbf{Postcondizioni:}} Il sistema ha creato correttamente il cliente desiderato e l'attore ha ricevuto un messaggio di conferma del successo dell'operazione.\\
\textsc{\textbf{Estensioni:}} L'attore invia una richiesta non valida (\hyperref[sec:UC16]{UC16}).

\subsubsection{UC13.4: Rimozione cliente}
\label{sec:UC134}
\textsc{\textbf{Attori:}} Front end.\\
\textsc{\textbf{Scopo e descrizione:}} L'attore vuole eliminare un cliente per la farmacia corrente.\\
\textsc{\textsc{\textbf{Precondizioni:}}} L'attore è collegato al sistema e possiede una sessione attiva per un utente con ruolo ``pharmacy''.\\
\textsc{\textbf{Scenario principale:}} L'attore richiede al sistema di eliminare il cliente desiderato fornendone il codice identificativo interno.\\
\textsc{\textbf{Postcondizioni:}} Il sistema ha rimosso correttamente il cliente desiderato e l'attore ha ricevuto un messaggio di conferma del successo dell'operazione.\\
\textsc{\textbf{Estensioni:}} L'attore invia una richiesta non valida (\hyperref[sec:UC16]{UC16}).

\subsubsection{UC13.5: Aggiornamento dati cliente}
\label{sec:UC135}
\textsc{\textbf{Attori:}} Front end.\\
\textsc{\textbf{Scopo e descrizione:}} L'attore vuole aggiornare i dati di un cliente della farmacia corrente.\\
\textsc{\textsc{\textbf{Precondizioni:}}} L'attore è collegato al sistema e possiede una sessione attiva per un utente con ruolo ``pharmacy''.\\
\textsc{\textbf{Scenario principale:}} L'attore invia al sistema una richiesta di aggiornamento dei dati del cliente desiderato fornendone il codice identificativo interno e i nuovi dati da inserire.\\
\textsc{\textbf{Postcondizioni:}} Il sistema ha aggiornato correttamente il cliente desiderato e l'attore ha ricevuto un messaggio di conferma del successo dell'operazione.\\
\textsc{\textbf{Estensioni:}} L'attore invia una richiesta non valida (\hyperref[sec:UC16]{UC16}).

\subsubsection{UC13.6: Invio codice OTP}
\label{sec:UC136}
\textsc{\textbf{Attori:}} Front end.\\
\textsc{\textbf{Scopo e descrizione:}} L'attore vuole inviare il codice per l'accettazione della privacy policy ad un cliente della farmacia corrente.\\
\textsc{\textsc{\textbf{Precondizioni:}}} L'attore è collegato al sistema e possiede una sessione attiva per un utente con ruolo ``pharmacy'', il cliente possiede un numero di cellulare valido.\\
\textsc{\textbf{Scenario principale:}} 
\begin{itemize}
    \item L'attore richiede al sistema l'invio del codice al cliente desiderato fornendone il codice identificativo interno e i nuovi dati da inserire;
    \item Il sistema richiede l'invio di un SMS con il codice desiderato ad Amazon \ref{itm:sns}.
\end{itemize}
\textsc{\textbf{Postcondizioni:}} Il sistema ha inoltrato correttamente la richiesta di invio e l'attore ha ricevuto un messaggio di conferma del successo dell'operazione.\\
\textsc{\textbf{Estensioni:}} \begin{itemize}
    \item L'attore invia una richiesta non valida (\hyperref[sec:UC16]{UC16});
    \item L'attore effettua troppi tentativi di invio (\hyperref[sec:UC17]{UC17}).
\end{itemize}

\subsubsection{UC13.7: Verifica codice OTP}
\label{sec:UC137}
\textsc{\textbf{Attori:}} Front end.\\
\textsc{\textbf{Scopo e descrizione:}} L'attore vuole verificare il codice per l'accettazione della privacy policy da parte di un cliente della farmacia corrente.\\
\textsc{\textsc{\textbf{Precondizioni:}}} L'attore è collegato al sistema e possiede una sessione attiva per un utente con ruolo ``pharmacy'', il cliente ha ricevuto il codice OTP.\\
\textsc{\textbf{Scenario principale:}} \begin{itemize}
    \item L'attore richiede al sistema la verifica del codice del cliente desiderato fornendo il codice OTP e il codice identificativo interno del cliente.
    \item Il sistema verifica la validità del codice.
\end{itemize}
\textsc{\textbf{Postcondizioni:}} Il sistema ha registrato l'accettazione della privacy policy, e l'attore ha ricevuto un messaggio di conferma del successo dell'operazione.\\
\textsc{\textbf{Estensioni:}} L'attore invia un codice non valido (\hyperref[sec:UC16]{UC16}).

\subsubsection{UC13.8: Caricamento manuale della \textit{privacy policy}}
\label{sec:UC138}
\textsc{\textbf{Attori:}} Front end.\\
\textsc{\textbf{Scopo e descrizione:}} L'attore vuole aggiungere al record di un cliente della farmacia una copia del modulo per il consenso al trattamento dei dati, firmato dal cliente .\\
\textsc{\textsc{\textbf{Precondizioni:}}} L'attore è collegato al sistema e possiede una sessione attiva per un utente con ruolo ``pharmacy'', esiste una copia digitale del modulo firmato.\\
\textsc{\textbf{Scenario principale:}} L'attore invia al sistema una richiesta di registrazione dell'accettazione della privacy policy, fornendo la copia digitale del modulo e il codice identificativo interno del cliente.\\
\textsc{\textbf{Postcondizioni:}} Il sistema ha registrato correttamente l'accettazione della \textit{privacy policy}, ha caricato correttamente la copia del modulo e l'attore ha ricevuto un messaggio di conferma del successo dell'operazione.\\
\textsc{\textbf{Estensioni:}} L'attore invia una richiesta non valida (\hyperref[sec:UC16]{UC16}).
    
\subsection{Casi d'uso secondari per UC14}\subsubsection{UC14.1: Richiesta informazioni lista delle spedizioni}
\label{sec:UC141}
\textsc{\textbf{Attori:}} Front end.\\
\textsc{\textbf{Scopo e descrizione:}} L'attore vuole ottenere la lista delle spedizioni per la farmacia corrente.\\
\textsc{\textsc{\textbf{Precondizioni:}}} L'attore è collegato al sistema e possiede una sessione attiva per un utente con ruolo ``pharmacy''.\\
\textsc{\textbf{Scenario principale:}} L'attore richiede al sistema i dati delle spedizioni appartenenti alla farmacia corrente.
\textsc{\textbf{Postcondizioni:}} L'attore ha ricevuto correttamente i dati richiesti al sistema.\\
\textsc{\textbf{Estensioni:}} L'attore invia una richiesta non valida (\hyperref[sec:UC16]{UC16}).\\

\subsubsection{UC14.2: Richiesta informazioni spedizione}
\label{sec:UC142}
\textsc{\textbf{Attori:}} Front end.\\
\textsc{\textbf{Scopo e descrizione:}} L'attore vuole ottenere i dati di una spedizione per la farmacia corrente.\\
\textsc{\textsc{\textbf{Precondizioni:}}} L'attore è collegato al sistema e possiede una sessione attiva per un utente con ruolo ``pharmacy''.\\
\textsc{\textbf{Scenario principale:}} L'attore richiede al sistema i dati di una spedizione appartenente alla farmacia corrente fornendone il codice identificativo interno.
\textsc{\textbf{Postcondizioni:}} L'attore ha ricevuto correttamente i dati richiesti al sistema.\\
\textsc{\textbf{Estensioni:}} L'attore invia una richiesta non valida (\hyperref[sec:UC16]{UC16}).

\subsubsection{UC14.3: Creazione spedizione}
\label{sec:UC144}
\textsc{\textbf{Attori:}} Front end.\\
\textsc{\textbf{Scopo e descrizione:}} L'attore vuole creare una spedizione per la farmacia corrente.\\
\textsc{\textsc{\textbf{Precondizioni:}}} L'attore è collegato al sistema e possiede una sessione attiva per un utente con ruolo ``pharmacy''.\\
\textsc{\textbf{Scenario principale:}} L'attore invia al sistema una richiesta di creazione di una spedizione contenente i dati necessari e una lista di campioni da spedire.
\textsc{\textbf{Postcondizioni:}} Il sistema ha creato correttamente la spedizione desiderata e l'attore ha ricevuto un messaggio di conferma del successo dell'operazione.
\textsc{\textbf{Estensioni:}} L'attore invia una richiesta non valida (\hyperref[sec:UC16]{UC16}).

\subsubsection{UC14.4: Rimozione spedizione}
\label{sec:UC144}
\textsc{\textbf{Attori:}} Front end.\\
\textsc{\textbf{Scopo e descrizione:}} L'attore vuole eliminare una spedizione per la farmacia corrente.\\
\textsc{\textsc{\textbf{Precondizioni:}}} L'attore è collegato al sistema e possiede una sessione attiva per un utente con ruolo ``pharmacy''.\\
\textsc{\textbf{Scenario principale:}} L'attore richiede al sistema di eliminare la spedizione desiderata fornendone il codice identificativo interno.
\textsc{\textbf{Postcondizioni:}} Il sistema ha rimosso correttamente la spedizione desiderata e l'attore ha ricevuto un messaggio di conferma del successo dell'operazione.
\textsc{\textbf{Estensioni:}} L'attore invia una richiesta non valida (\hyperref[sec:UC16]{UC16}).

\subsubsection{UC14.5: Aggiornamento dati spedizione}
\label{sec:UC145}
\textsc{\textbf{Attori:}} Front end.\\
\textsc{\textbf{Scopo e descrizione:}} L'attore vuole aggiornare i dati di un cliente della farmacia corrente.\\
\textsc{\textsc{\textbf{Precondizioni:}}} L'attore è collegato al sistema e possiede una sessione attiva per un utente con ruolo ``pharmacy''.\\
\textsc{\textbf{Scenario principale:}} L'attore invia al sistema una richiesta di aggiornamento dei dati della spedizione desiderata fornendone il codice identificativo interno e i nuovi dati da inserire.
\textsc{\textbf{Postcondizioni:}} Il sistema ha aggiornato correttamente la spedizione desiderata e l'attore ha ricevuto un messaggio di conferma del successo dell'operazione.
\textsc{\textbf{Estensioni:}} L'attore invia una richiesta non valida (\hyperref[sec:UC16]{UC16}).

\subsubsection{UC14.6: Visualizzazione lettera di vettura}
\label{sec:UC146}
\textsc{\textbf{Attori:}} Front end.\\
\textsc{\textbf{Scopo e descrizione:}} L'attore vuole visualizzare la lettera di vettura di una spedizione per la farmacia corrente.\\
\textsc{\textsc{\textbf{Precondizioni:}}} L'attore è collegato al sistema e possiede una sessione attiva per un utente con ruolo ``pharmacy'', i dati della spedizione sono validi.\\
\textsc{\textbf{Scenario principale:}} 
\begin{itemize}
    \item L'attore richiede al sistema la lettera di vettura per la spedizione desiderata fornendone il codice identificativo interno;
    \item Il sistema ottiene la lettera di vettura dalle \ref{itm:api} SDA;
    \item Il sistema invia la lettera di vettura all'attore.
\end{itemize}
\textsc{\textbf{Postcondizioni:}} L'attore visualizza correttamente la lettera di vettura.\\
\textsc{\textbf{Estensioni:}} L'attore invia una richiesta non valida (\hyperref[sec:UC16]{UC16}).

\subsubsection{UC14.7: Prenotazione spedizione}
\label{sec:UC147}
\textsc{\textbf{Attori:}} Front end.\\
\textsc{\textbf{Scopo e descrizione:}} L'attore vuole prenotare una spedizione per la farmacia corrente presso SDA.\\
\textsc{\textsc{\textbf{Precondizioni:}}} L'attore è collegato al sistema e possiede una sessione attiva per un utente con ruolo ``pharmacy'', i dati della spedizione sono validi.\\
\textsc{\textbf{Scenario principale:}} 
\begin{itemize}
    \item L'attore richiede al sistema la prenotazione della spedizione desiderata fornendone il codice identificativo interno;
    \item Il sistema effettua la prenotazione attraverso le \ref{itm:api} SDA.
\end{itemize}
\textsc{\textbf{Postcondizioni:}} Il sistema ha registrato il codice di prenotazione della spedizione e la data di ritiro, e l'attore ha ricevuto i dati aggiornati della spedizione e un messaggio di conferma del successo dell'operazione.\\
\textsc{\textbf{Estensioni:}} L'attore invia una richiesta non valida (\hyperref[sec:UC16]{UC16});
