%!TEX root = ../dissertation.tex
\chapter{Introduzione}
\label{introduction}
\paragraph{}
Il presente documento espone il lavoro svolto dal laureando Andrea Trevisin durante il periodo di stage formativo presso Moku S.r.l. al fine di realizzare il \textit{back end} dell'applicazione web ``Biota'', una piattaforma per la gestione di esami della flora gastrointestinale.

Lo stage, della durata di 300 (trecento) ore, prevedeva il raggiungimento di diversi obiettivi: in primis, lo sviluppo di un backend in Ruby on Rails 5 completo delle funzionalità base per completare il flusso di operazioni da parte della farmacia e di API GraphQL per l'interazione con il \textit{front end}; in secondo luogo era prevista l'implementazione di API REST utilizzabile dal laboratorio di analisi, l'integrazione con le API del corriere SDA per la prenotazione di spedizioni e la realizzazione di una suite di test completa; infine sarebbe avvenuta l'implementazione di requisiti facoltativi, tra cui la generazione dinamica del referto e l'invio di codici OTP tramite SMS.

Al termine della durata dello stage tutti gli obiettivi possono dirsi pienamente soddisfatti.

\section{Azienda}
\paragraph{}
Moku S.r.l. è una startup nata nel 2013, con sede a Roncade (TV), fondata su un progetto software omonimo per una piattaforma web mirata a facilitare il lavoro condiviso su documenti di vario tipo. Si è poi evoluta diventando una società di consulenza IT che realizza prodotti software su commissione. 
Il team di Moku usa metodologie agili, basate su Scrum, e lavora a stretto contatto con il cliente; questo permette di individuare facilmente e velocemente i requisiti del prodotto, creando prodotti di qualità che rispecchiano le esigenze del committente. I principali ambiti di sviluppo di Moku sono piattaforme web applicazioni Android/IOS.

\section{Processi aziendali}
\subsection{Modello di ciclo di vita del software}
\paragraph{}
La metodologia di sviluppo utilizzata dal team di Moku si basa su Scrum, un \textit{framework} agile per la gestione del ciclo di sviluppo del software, iterativo ed incrementale, creato e sviluppato da Ken Schwaber e Jeff Sutherland. Scrum si presta bene alla modalità di lavoro dell'azienda, in quanto è ideato per team di dimensioni ridotte e permette di gestire facilmente progetti anche complessi, in cui è difficile effettuare una pianificazione esaustiva, offrendo la possibilità di reagire velocemente a cambiamenti: una delle idee chiave è infatti la "volatilità dei requisiti", ovvero il riconoscere che le esigenze dei clienti possano variare in corso d'opera, e che possano sorgere complicazioni non previste - evento comune quando si usano tecnologie all'avanguardia. L'obiettivo principale del metodo Scrum è quindi la realizzazione di un prodotto software funzionante e soddisfacente a scapito di aspetti ritenuti non essenziali nell'immediato (e.g. una documentazione completa). 
Alla base di Scrum, dunque, c'è la teoria dei controlli empirici dell'analisi strumentale e funzionale di processo

