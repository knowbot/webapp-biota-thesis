%!TEX root = ../dissertation.tex
\externaldocument{../frontmatter/abbr}
\begin{savequote}[75mm]
Nulla facilisi. In vel sem. Morbi id urna in diam dignissim feugiat. Proin molestie tortor eu velit. Aliquam erat volutpat. Nullam ultrices, diam tempus vulputate egestas, eros pede varius leo.
\qauthor{Quoteauthor Lastname}
\end{savequote}

\chapter{Verifica e validazione}
\section{Premessa}
L'applicativo sviluppato nel corso di questo progetto, come peraltro già detto, è entrato in fase di produzione (ovvero è stato rilasciato per l'uso da parte del cliente) relativamente presto, con aggiornamenti regolari che introducevano man mano nuove funzionalità. Tale situazione ha introdotto la forte necessità di controllare che ogni versione rilasciata del prodotto adempiesse correttamente a tutti i bisogni del cliente e fosse priva di inadeguatezze; il che ha reso ancora più fondamentali la verifica e la validazione del codice. In questa sezione verranno esposte le convenzioni e i processi volti ad assicurare la qualità del prodotto, espressa in assenza di errori introdotti da nuove funzionalità e adempimento ai bisogni espressi dal cliente.
\section{Norme di codifica}
Il primo passo nella realizzazione di un prodotto di qualità consiste nell'adottare norme che minimizzino i rischi e gli errori che si possono verificare. Sono quindi state seguite delle precise norme di codifica volte a rendere il codice funzionale e facilmente comprensibile a sviluppatori diversi dall'autore.

Sono quindi state seguite pedissequamente le norme di codifica stabilite dalla community di Ruby on Rails, raccolte in una \textit{style guide} pubblicamente disponibile.
\section{Verifica}
\subsection{Analisi statica}
\subsection{Analisi dinamica}
\section{Struttura dei test}
\subsection{Test di unità}
\subsection{Test di integrazione}
\section{Validazione}


